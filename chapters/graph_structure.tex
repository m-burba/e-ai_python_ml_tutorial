\documentclass{article}
\usepackage{tikz}
\usepackage{amsmath}
\usepackage{amssymb} % <-- Add this
\usetikzlibrary{positioning, arrows.meta, shapes.geometric}
\usepackage[active,tightpage]{preview}
\PreviewEnvironment{tikzpicture}

\begin{document}

\begin{center}
\begin{preview}
\begin{tikzpicture}[
    scale=0.7, transform shape,
    node distance=1.6cm,
    every node/.style={font=\sffamily},
    box/.style={draw, minimum height=1.0cm, minimum width=2.0cm, align=center},
    arrow/.style={-Stealth, thick},
    round/.style={circle, draw, minimum size=7mm}
]

% Nodes (left to right)
\node[box] (input) {Node\\features};
\node[box, right=of input] (inputproj) {Linear\\projection};
\node[box, right=of inputproj] (gnn1) {GNN layer\\$\mathbf{W}_{\text{self}} + \mathbf{W}_{\text{neigh}}$};
\node[box, right=of gnn1] (gnn2) {GNN layer\\$\mathbf{W}_{\text{gnn}} + \mathbf{W}_{\text{neigh}}$};
\node[box, right=of gnn2] (output) {Output};

% Arrows
\draw[arrow] (input) -- (inputproj);
\draw[arrow] (inputproj) -- (gnn1);
\draw[arrow] (gnn1) -- (gnn2);
\draw[arrow] (gnn2) -- (output);

% Input feature vector
\node[round, above=0.8cm of input] (featurevec) {\tiny$\begin{bmatrix} 0 \\ 1 \end{bmatrix}$};
\draw[arrow] (featurevec) -- (input);

% Graph neighborhood on top of both GNNs
\node[round, above=1.6cm of gnn1, xshift=1.6cm] (node1) {2};
\node[round, above left=0.6cm and 0.8cm of node1] (node2) {};
\node[round, above right=0.6cm and 0.8cm of node1] (node3) {};
\draw[arrow] (node2) -- (node1);
\draw[arrow] (node3) -- (node1);
\draw[arrow] (node1) -- ++(0,-0.8) -- ++(-1.6,0) -- (gnn1.north);
\draw[arrow] (node1) -- ++(0,-0.8) -- ++(1.6,0) -- (gnn2.north);

% Comments with matrix sizes
\node[below=0.4cm of inputproj] {\tiny$\mathbf{W}_\text{in} \in \mathbb{R}^{64 \times 3},\ \mathbf{b}_\text{in} \in \mathbb{R}^{64}$};
\node[below=0.4cm of gnn1] {\tiny$\mathbf{W}_\text{self} \in \mathbb{R}^{64 \times 64},\ \mathbf{W}_\text{neigh} \in \mathbb{R}^{64 \times 64}$};
\node[below=0.4cm of gnn2] {\tiny$\mathbf{W}_\text{self} \in \mathbb{R}^{64 \times 64},\ \mathbf{W}_\text{neigh} \in \mathbb{R}^{64 \times 64}$};
\node[below=0.4cm of output] {\tiny$\mathbf{W}_\text{out} \in \mathbb{R}^{1 \times 64},\ \mathbf{b}_\text{out} \in \mathbb{R}^{1}$};

\end{tikzpicture}
\end{preview}
\end{center}

\end{document}

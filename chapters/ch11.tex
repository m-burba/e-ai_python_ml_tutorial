
\chapter{Agents and Coding with LLM}

\section{Introduction to Automated Coding with LLM}
Large Language Models (LLMs) can be leveraged to assist in writing code, generating scripts, and automating tasks. This section introduces a simple example where we generate Python code, save it, and execute it dynamically.

\subsection{Generating and Executing Code from an LLM}
To interact with an LLM and generate code, we can use OpenAI’s API. Below is an example of how to generate, save, and execute Python code.

\begin{codeonly}{Generating and Running Python Code from LLM}
import openai
import os

def get_code_from_llm(prompt):
    client = openai.Client(api_key=os.getenv("OPENAI_API_KEY"))
    response = client.chat.completions.create(
        model="gpt-4o-mini",
        messages=[
            {"role": "system", "content": "You are an AI coder. Provide only executable Python code."},
            {"role": "user", "content": prompt}
        ]
    )
    return response.choices[0].message.content.strip()

code = get_code_from_llm("Write a Python function that computes the Fibonacci sequence up to n.")

# Save code to a file
with open("generated_script.py", "w") as f:
    f.write(code)

# Execute the script
exec(open("generated_script.py").read())
\end{codeonly}

\section{Survey of Agent Frameworks}
There are several agent-based frameworks that integrate LLMs for code execution and automation. Two common ones include:

\begin{itemize}
\item
{\bf LangChain}: A framework designed for building applications with LLMs that integrate memory, reasoning, and chaining capabilities.
\item
{\bf Auto-GPT}: An autonomous agent framework that can self-prompt, plan, and execute tasks using LLMs.
\end{itemize}

\section{Example 1: Using LangChain for Code Execution}
LangChain provides tools to let LLMs execute tasks dynamically, such as writing and running Python code.

\subsection{Installation}
Install LangChain and OpenAI API:

\begin{codeonly}{Installing LangChain}
!pip install langchain openai
\end{codeonly}

\subsection{Using LangChain to Automate Code Execution}

\begin{codeonly}{LangChain Script for Automated Coding}
from langchain.llms import OpenAI
from langchain.chains import LLMChain
from langchain.prompts import PromptTemplate
import openai

# Define the prompt template
prompt_template = PromptTemplate.from_template("""
Write a Python script that fetches a 2m temperature field from DWD open data and plots it.
""")

# Initialize the LLM
llm = OpenAI(api_key=os.getenv("OPENAI_API_KEY"))
chain = LLMChain(llm=llm, prompt=prompt_template)

# Get generated code
code = chain.run("")

# Save and execute
with open("generated_weather_script.py", "w") as f:
    f.write(code)
exec(open("generated_weather_script.py").read())
\end{codeonly}

\section{Example 2: Using Auto-GPT to Automate Tasks}
Auto-GPT is an advanced framework for autonomous coding agents. We set it up and use it to generate and run Python scripts.

\subsection{Installation}

\begin{codeonly}{Installing Auto-GPT}
!git clone https://github.com/Torantulino/Auto-GPT.git
!cd Auto-GPT && pip install -r requirements.txt
\end{codeonly}

\subsection{Using Auto-GPT to Generate and Execute Code}

\begin{codeonly}{Auto-GPT Code Execution}
from autogpt.agent import AutoGPT

auto_gpt = AutoGPT(api_key=os.getenv("OPENAI_API_KEY"))

response = auto_gpt.run_task("Fetch a 2m temperature field from DWD open data and display it with Matplotlib and Cartopy.")

print("Generated Code:")
print(response)
\end{codeonly}

\section{Generated Code: Fetching and Plotting a 2m Temperature Field}
Once the agent generates the code, we execute it to visualize the data.

\begin{codeonly}{Fetching and Plotting 2m Temperature from DWD}
import eccodes
import numpy as np
import matplotlib.pyplot as plt
import cartopy.crs as ccrs
import requests
import os

# Download GRIB file
url = "https://opendata.dwd.de/weather/nwp/icon-eu/grib/00/t_2m.grib2"
response = requests.get(url)
with open("temperature.grib2", "wb") as f:
    f.write(response.content)

# Read GRIB data
with open("temperature.grib2", "rb") as f:
    gid = eccodes.codes_grib_new_from_file(f)
    values = eccodes.codes_get_values(gid)
    latitudes = eccodes.codes_get_array(gid, "latitudes")
    longitudes = eccodes.codes_get_array(gid, "longitudes")
    eccodes.codes_release(gid)

# Reshape data
grid_shape = (int(np.sqrt(len(values))), int(np.sqrt(len(values))))
values = values.reshape(grid_shape)
latitudes = latitudes.reshape(grid_shape)
longitudes = longitudes.reshape(grid_shape)

# Plot temperature field
plt.figure(figsize=(10, 6))
ax = plt.axes(projection=ccrs.PlateCarree())
plt.contourf(longitudes, latitudes, values, cmap="coolwarm")
plt.colorbar(label="2m Temperature (degreeC)")
ax.coastlines()
plt.title("2m Temperature from DWD Open Data")
plt.show()
\end{codeonly}

This section demonstrates how to use LLMs and agent frameworks to generate, execute, and visualize weather data from DWD OpenData.

\section{Building a Vector Database as Package}

We'll structure your package as follows:

\begin{verbatim}
faiss_query_tool/
|-- faiss_query_tool/          # Main package folder
|   |-- code-ch01-sec02-code-ch01-sec02-__init__.py
|   |-- code-ch14-sec01-code-ch14-sec01-loader.py              # Handles document loading
|   |-- code-ch14-sec01-code-ch14-sec01-embeddings.py          # Handles embedding model
|   |-- code-ch14-sec01-code-ch14-sec01-faiss_index.py         # Handles FAISS index
|   |-- code-ch14-sec01-code-ch14-sec01-query.py               # Querying functionality
|   |-- code-ch14-sec01-code-ch14-sec01-openai_chat.py         # OpenAI integration
|-- tests/                     # Test scripts
|-- code-ch01-sec02-code-ch01-sec02-setup.py                   # Package metadata and installation
|-- README.md                  # Documentation
|-- requirements.txt           # Required dependencies
\end{verbatim}


\subsection{Load files into Vector Database}

We first build a loader

%\includeexternalcode{faiss\_query\_tool/code\_files/code-ch14-sec01-faiss-vector-db.py}{faiss_query_tool/code_files/code-ch14-sec01-faiss-vector-db.py}


%\includeexternalcode{faiss\_query\_tool/code\_files/code-ch14-sec01-generate-text-embedding.py}{faiss_query_tool/code_files/code-ch14-sec01-generate-text-embedding.py}


%\includeexternalcode{faiss\_query\_tool/code\_files/code-ch14-sec01-faiss-vector-db.py}{faiss_query_tool/code_files/code-ch14-sec01-faiss-vector-db.py}


%\includeexternalcode{faiss\_query\_tool/code\_files/code-ch14-sc01-faiss-vector-db.py}{faiss_query_tool/code_files/code-ch14-sec01-faiss-vector-db.py}


%\includeexternalcode{faiss\_query\_tool/code\_files/code-ch14-sec01-faiss-vector-db.py}{faiss_query_tool/code_files/code-ch14-sec01-faiss-vector-db.py}


%\includeexternalcode{faiss\_query\_tool/code\_files/code-ch14-sec01-faiss-vector-search-tool.py}{faiss_query_tool/code_files/code-ch14-sec01-faiss-vector-search-tool.py}


\subsection{Installation of your package}

In the root directory of your package, run

\begin{codeonly}{Installation}
pip install .
\end{codeonly}

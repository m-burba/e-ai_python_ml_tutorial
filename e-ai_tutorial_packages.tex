
% =============================================
%  Encoding and Fonts
% =============================================
\usepackage[utf8]{inputenc} % UTF-8 encoding for compatibility
\usepackage{lmodern}        % Latin Modern for improved scaling and clarity
\usepackage{microtype}      % Enhances spacing and readability
\usepackage{amsmath, amssymb}

% ---- Sans-Serif Font (Helvetica) ----
\usepackage{helvet}              % Use Helvetica for a modern sans-serif look
\renewcommand{\familydefault}{\sfdefault}  % Set sans-serif as the default font

% ---- Monospace Font (Fira Code) ----
\usepackage[T1]{fontenc}  % Ensure proper font encoding
\usepackage{FiraMono}      % Use Fira Mono for monospaced text
%\renewcommand{\ttdefault}{firamono} % Apply Fira Code globally to monospace
\renewcommand{\ttdefault}{zi4} % Apply Fira Code globally to monospace

% =============================================
%  Graphics and Colors
% =============================================
\usepackage{graphicx} % Enables image handling
\usepackage[table]{xcolor}
\usepackage{colortbl}        % explicitly load colortbl for table colors
\usepackage{longtable}

% fixing latex3 bug: 
%\makeatletter
%\providecommand{\g__tbl_row_int}{0}
%\makeatother

% ---- Define DWD-Inspired Colors ----
\definecolor{DWDblue}{RGB}{0,68,136}     % Primary DWD blue
\definecolor{dwdspecial}{RGB}{50,30,80}  % Slightly lighter blue for contrast
\definecolor{dwdblue}{RGB}{0,68,136}     % Alternative DWD blue (identical to primary)
% Define Colors
\definecolor{codegreen}{rgb}{0.0, 0.5, 0.0} % Dark green for comments
\definecolor{codeblue}{rgb}{0.0, 0.2, 0.6} % Muted dark blue for keywords
\definecolor{codered}{rgb}{0.6, 0.1, 0.1}  % Muted red for strings
\definecolor{codegray}{rgb}{0.4, 0.4, 0.4} % Gray for numbers and operators
\definecolor{backcolor}{rgb}{0.97, 0.97, 0.97} % Very light gray background
%\definecolor{DWDblue}{rgb}{0, 0.35, 0.7} % Define DWDblue
% Define a light blue color
\definecolor{lightblue}{RGB}{220, 230, 241} % Light blue for alternating rows
\definecolor{headerblue}{RGB}{180, 200, 230} % Slightly darker blue for headers
\definecolor{appendixblue}{RGB}{200, 220, 240} % Blue for appendix header


% --- Page Geometry ---
\usepackage[margin=1in]{geometry}

% --- Fancy Headers ---
\usepackage{fancyhdr}
\pagestyle{fancy}
\fancyhf{}
\fancyhead[L]{\nouppercase{\leftmark}}
\fancyhead[R]{\nouppercase{\rightmark}}
\fancyfoot[C]{\thepage}
% Update chapter and section marks for header display
\renewcommand{\chaptermark}[1]{\markboth{Chapter \thechapter: #1}{}}
\renewcommand{\sectionmark}[1]{\markright{Section \thesection: #1}}

% --- Section and Chapter Title Formatting ---
\usepackage{titlesec}
\titleformat{\chapter}[display]
  {\normalfont\Huge\bfseries\color{DWDblue}}
  {\chaptertitlename\ \thechapter}{20pt}{\Huge}
  
\titleformat{\section}
  {\normalfont\Large\bfseries\color{DWDblue}}
  {\thesection}{1em}{}

%\usepackage[utf8]{inputenc}

% --- For Dummy Text ---
\usepackage{lipsum}

\fancyhead[R]{\includegraphics[width=0.2\textwidth]{images/DWD-Logo_2013.svg.png}}
\setlength{\headsep}{40pt}  % Adjust the value as needed

\usepackage{listings}
\usepackage[listings, breakable]{tcolorbox} % Enable listings inside tcolorbox
%\usepackage{xcolor}
\usepackage{accsupp}

\lstdefinelanguage{Markdown}{
  morekeywords={[1]\#,\#\#,\#\#\#,\#\#\#\#},
  morekeywords={[2]*,**,---,-},
  sensitive=false,
  morecomment=[l][\color{gray}]{<!--},
  morestring=[b]",
}

\lstdefinestyle{mdstyle}{
  language=Markdown,
  basicstyle=\ttfamily\small,
  breaklines=true,
  keywordstyle={[1]\color{blue}\bfseries},
  keywordstyle={[2]\color{purple}},
  commentstyle=\color{gray},
  stringstyle=\color{orange},
}

% Define the Python syntax highlighting style
\lstdefinestyle{pythonstyle}{
    language=Python,
    basicstyle=\ttfamily\small,    % Monospace font
    keywordstyle=\color{codeblue}\bfseries, % Keywords in blue
    stringstyle=\color{codered},  % Strings in red
    commentstyle=\color{codegreen}\itshape, % Comments in green italics
    numberstyle=\tiny\color{codegray}, % Line numbers in gray
    backgroundcolor=\color{gray!5}, % Light gray background
    breaklines=true,
    columns=fixed,
    showspaces=false,
    showstringspaces=false,
    keepspaces=true,
    numbers=left, % Show line numbers
    numbersep=5pt, % Adjust separation between numbers and text
    xleftmargin=7pt, % **Fix: Keeps numbers inside the box**
    framexleftmargin=15pt, % **Fix: Adds padding inside the box for numbers**
}

\lstdefinestyle{outputstyle}{
    language=Python,
    basicstyle=\ttfamily\small,    % Monospace font
    keywordstyle=\color{codeblue}, % Dark blue for keywords
    stringstyle=\color{codered},   % Muted red for strings
    commentstyle=\color{codegreen}\itshape, % Dark green italic for comments
    numberstyle=\color{codegray},  % Gray for numbers
    backgroundcolor=\color{backcolor}, % Light gray background
    breaklines=true,
    columns=fullflexible,
    showspaces=false,
    showstringspaces=false,
    keepspaces=true,
    xleftmargin=5pt, % Moves the entire box to the right
    framexleftmargin=5pt, % Keeps the frame aligned inside the box
}

% Define the Python syntax highlighting style
\lstdefinestyle{pythonstyle2}{
    language=Python,
    basicstyle=\ttfamily\small,    % Monospace font
    keywordstyle=\color{codeblue}\bfseries, % Keywords in blue
    stringstyle=\color{codered},  % Strings in red
    commentstyle=\color{codegreen}\itshape, % Comments in green italics
    numberstyle=\tiny\color{codegray}, % Line numbers in gray
    backgroundcolor=\color{gray!5}, % Light gray background
    breaklines=true,
    columns=fixed,
    showspaces=false,
    showstringspaces=false,
    keepspaces=true,
    numbers=none, % Show line numbers
    numbersep=5pt, % Adjust separation between numbers and text
    xleftmargin=7pt, % **Fix: Keeps numbers inside the box**
    framexleftmargin=15pt, % **Fix: Adds padding inside the box for numbers**
}


\newcommand{\includeexternalcode}[2]{%
  \begin{tcolorbox}[breakable,
    colback=gray!5,                        % Soft light gray background
    colframe=white,                      % Light gray thin border
    frame style={line width=0.2pt, draw=gray!30}, % Ultra-thin frame
    borderline west={0.2pt}{0pt}{DWDblue!50}, % Optional subtle blue accent on the left
    arc=1.5mm,                              % Smooth rounded corners
    left=2mm, right=2mm, top=1mm, bottom=1mm, % Modern compact spacing
    before skip=8pt, after skip=8pt,         % Keeps spacing subtle
    coltitle=gray!100,                          % **Fix: Ensure title text is black**
    title={#1},                             % Box title
    fonttitle=\bfseries\footnotesize        % Small, clean, modern title
  ]
    \lstinputlisting[style=pythonstyle]{code/#2}
  \end{tcolorbox}%
}


\newcommand{\includeexternaltext}[2]{%
  \begin{tcolorbox}[breakable,
    colback=gray!5,
    colframe=white, % Use a slightly visible gray border
		coltitle=black
    title={#1},
    left=2mm,
    right=2mm,
    top=2mm,
    bottom=2mm,
    before skip=8pt, after skip=8pt,         % Keeps spacing subtle
    listing only,
    enhanced jigsaw % Ensures compatibility with lstlisting
  ]
    \lstinputlisting[style=pythonstyle]{code/#2}
  \end{tcolorbox}%
}


\newtcblisting{codeonly}[2][]{%
  listing only,
  listing options={style=pythonstyle, basicstyle=\ttfamily\fontsize{9pt}{13pt}\selectfont},
  breakable,
  colback=gray!5,
  colframe=gray!30,
  boxrule=0.2pt,
  arc=1.5mm,
  left=2mm, right=2mm, top=1mm, bottom=1mm,
  before skip=8pt, after skip=8pt,
  coltitle=gray!100,
  title={#2},
  fonttitle=\bfseries\footnotesize,
  #1
}


\newtcblisting{codeonlysmall}[2][]{%
  listing only,
  listing options={style=pythonstyle2, basicstyle=\ttfamily\scriptsize},
  breakable,
  colback=gray!5,
  colframe=gray!30,
  boxrule=0.2pt,
  arc=1.5mm,
  left=2mm, right=2mm, top=1mm, bottom=1mm,
  before skip=8pt, after skip=8pt,
  coltitle=gray!100,
  title={#2},
  fonttitle=\bfseries\scriptsize,
  #1
}



\newcommand{\runpythoncode}[1]{%
  %\immediate\write18{cmd /c "D:/ropy/Scripts/activate && python #1"}%
}\setlength{\parskip}{3pt}


\newtcolorbox{recommendationbox}[1][]{%
  colback=yellow!10,      % light yellow background
  colframe=yellow!50!black, % darker yellow frame
  fonttitle=\bfseries,     % bold title
  title=Recommendation,    % title text
  arc=4mm,                 % rounded corners
  auto outer arc,
  boxrule=1pt,
  left=2mm, right=2mm, top=2mm, bottom=2mm,
  breakable,
  #1
}

\usepackage{verbatimbox} % to capture verbatim content into a macro
\usepackage{accsupp}      % to override ActualText

\newsavebox{\codebox}

\usepackage{tikz}
\usetikzlibrary{matrix,positioning, arrows.meta, calc}
\usepackage{lmodern}  % Lightweight modern font


\DeclareRobustCommand{\bbb}[1]{\textbf{\textcolor{black}{#1}}}

\setlength{\parindent}{0pt}
\setlength{\parskip}{5pt}


\usepackage[
  colorlinks=true,
  linkcolor=dwdblue,
  urlcolor=dwdblue,
  citecolor=dwdblue,
  pdfborder={0 0 1}
]{hyperref}

\newcommand{\mytext}[1]{\mbox{#1}}



% Required packages
\usepackage{tocloft}  % For styling Table of Contents
\usepackage{titlesec} % For section title formatting

% Customize Table of Contents Title
\renewcommand{\contentsname}{\color{dwdblue}Table of Contents}
\renewcommand{\cfttoctitlefont}{\color{dwdblue}\Huge\bfseries} % ToC title color

% Customize ToC Entries
\renewcommand{\cftchapfont}{\color{dwdblue}\bfseries}  % Chapter entries
\renewcommand{\cftsecfont}{\color{dwdblue}}            % Section entries
\renewcommand{\cftsubsecfont}{\color{dwdblue}}         % Subsection entries

% Customize Chapter Titles in Document
\titleformat{\chapter}[display]
  {\normalfont\Huge\bfseries\color{dwdblue}}{\chaptername\ \thechapter}{20pt}{\Huge}

\setcounter{tocdepth}{1}

\usepackage{enumitem}

\lstset{style=outputstyle} % Set pythonstyle as the default for lstinputlisting
